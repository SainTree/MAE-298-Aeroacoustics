\documentclass[onecolumn,10pt]{jhwhw}

\usepackage{epsfig} %% for loading postscript figures
\usepackage{amsmath}
\usepackage{graphicx}
\usepackage{grffile}
\usepackage{pdfpages}
\usepackage{algpseudocode}
\usepackage{wrapfig}
\usepackage{booktabs}
\usepackage{multicol}
\usepackage{cancel}

% Default fixed font does not support bold face
\DeclareFixedFont{\ttb}{T1}{txtt}{bx}{n}{12} % for bold
\DeclareFixedFont{\ttm}{T1}{txtt}{m}{n}{12}  % for normal

% Custom colors
\usepackage{color}
\usepackage{listings}
\usepackage{framed}
\usepackage{caption}
\usepackage{bm}
\captionsetup[lstlisting]{font={small,tt}}

\author{John Karasinski}
\title{EAE 298 Aeroacoustics \\ Fall Quarter 2016 \\ Homework \#3}

\begin{document}
\maketitle

\problem{[50 points]}
The acoustic wave equation without considering the source is expressed as follows:
\begin{align*}
\dfrac{1}{c^2} \dfrac{\partial^2 p}{\partial t^2} - \nabla^2 p = 0
\end{align*}
We can define a new function $\widetilde{p}$ using the imbedding technique as follows:
\begin{align*}
\widetilde{p} = p, & \hspace{1em} f > 0 \\
\widetilde{p} = 0, & \hspace{1em} f < 0
\end{align*}
where $f=0$ describes the arbitrary moving body. Show that the wave equation whose sound is generated by an arbitrary moving body ($f=0$) can be expressed as follows:
\begin{align*}
\dfrac{1}{c^2} \dfrac{\partial^2 \widetilde{p}}{\partial t^2}
- \overline{\nabla}^2 \widetilde{p}
= - \left [ \dfrac{M_n}{c} \dfrac{\partial p}{\partial t} + p_n \right] \delta (f)
- \dfrac{1}{c} \dfrac{\partial}{\partial t} \left[ M_n p \delta(f) \right]
- \nabla \cdot \left[ p \vec{n} \delta (f) \right]
\end{align*}
where $\vec{n}$ is the unit normal vector on the surface and $p_n = \nabla p \cdot \vec{n}$. Now we can use the Green’s function of the wave equation in the unbounded space, the so-called free-space Green’s function, to find the unknown function $p(\vec{x}, t)$ everywhere in space. The result is the Kirchhoff formula for moving surfaces.

\solution
Using the rules of generalized differentiation, we obtain
\begin{align*}
\dfrac{1}{c} \dfrac{\overline{\partial} \widetilde{p}}{\partial t} =& \dfrac{1}{c} \dfrac{\partial \widetilde{p}}{\partial t} - M_n p \delta (f) \\
\dfrac{1}{c} \dfrac{\overline{\partial}^2 \widetilde{p}}{\partial t^2} =& \dfrac{1}{c} \dfrac{\partial^2 \widetilde{p}}{\partial t^2}
- \dfrac{M_n}{c} \dfrac{\partial p}{\partial t} \delta (f)
-\dfrac{1}{c} \dfrac{\partial}{\partial t} [M_n p \delta (f)]
\end{align*}
where $M_n = v_n/c$ is the local normal Mach number of the surface $f=0$. Similarly, we find
\begin{align*}
\overline{\nabla} \widetilde{p} =& \nabla \widetilde{p} + p \vec{n} \delta (f) \\
\overline{\nabla}^2 \widetilde{p} =& \nabla^2 \widetilde{p} + p_n \delta (f)
+ \nabla \cdot [p \vec{n} \delta (f)]
\end{align*}
Combining these equations yields
\begin{align*}
\dfrac{1}{c^2} \dfrac{\partial^2 \widetilde{p}}{\partial t^2}
- \overline{\nabla}^2 \widetilde{p}
&=
\dfrac{1}{c} \dfrac{\partial^2 \widetilde{p}}{\partial t^2}
- \dfrac{M_n}{c} \dfrac{\partial p}{\partial t} \delta (f)
-\dfrac{1}{c} \dfrac{\partial}{\partial t} [M_n p \delta (f)]
- \left( \nabla^2 \widetilde{p} + p_n \delta (f)
+ \nabla \cdot [p \vec{n} \delta (f)] \right) \\
&=
\cancelto{0}{
\left [\dfrac{1}{c} \dfrac{\partial^2 \widetilde{p}}{\partial t^2}  - \nabla^2 \widetilde{p} \right]}
- \left[ \dfrac{M_n}{c} \dfrac{\partial p}{\partial t} + p_n \delta (f)\right] \delta (f)
-\dfrac{1}{c} \dfrac{\partial}{\partial t} [M_n p \delta (f)]
- \nabla \cdot [p \vec{n} \delta (f)] \\
&=
- \left[ \dfrac{M_n}{c} \dfrac{\partial p}{\partial t} + p_n \delta (f)\right] \delta (f)
-\dfrac{1}{c} \dfrac{\partial}{\partial t} [M_n p \delta (f)]
- \nabla \cdot [p \vec{n} \delta (f)]
\end{align*}
where we have used the property that the original acoustic wave equation is equal to zero to remove the first two terms. Upon use of the Green's Function of the wave operator in unbounded space, this equation gives
\begin{align*}
4 \pi \widetilde{p}(\vec{x}, t) &=
- \int \dfrac{1}{r} \left( p_n + \dfrac{M_n}{c} p_{\tau(y)} \right) \delta (f) \delta(g) d \vec{y} d \tau
- \dfrac{1}{c} \dfrac{\overline{\partial}}{\partial t} \int \dfrac{1}{r} M_n p \delta (f) \delta(g) d \vec{y} d \tau \\
&- \overline{\nabla} \cdot \int \dfrac{1}{r} p \vec{n} \delta (f) \delta (g) d \vec{y} d \tau
\end{align*}
where $g=\tau - t + r/c$, and $p_{\tau (y)} = \partial p (\vec{y}, \tau) / \partial \tau$.

\problem{[50 points]}
Farassat’s formulation 1 for the loading noise is given as
\begin{align*}
4 \pi p_L^{\prime}(\vec{x}, t) = \dfrac{1}{c} \dfrac{\partial}{\partial t}
\int_{f=0} \left[ \dfrac{L_r}{r(1-M_r)} \right]_{ret} dS + \int_{f=0} \left[ \dfrac{L_r}{r^2(1-M_r)} \right]_{ret} dS
\end{align*}
where $L_r = \Delta P \vec{n} \cdot \hat{r} = \Delta P \cos \theta$. This formulation 1 is difficult to compute since the observer time differentiation is outside the integrals. A much more efficient and practical formulation can be derived by carrying the observer time derivate inside the integrals (formulation 1A). Show that formulation 1A for the loading noise becomes
\begin{align*}
4 \pi p_L^{\prime}(\vec{x}, t) =& \dfrac{1}{c} \int_{f=0} \left[ \dfrac{\dot{L_r}}{r(1-M_r)^2} \right]_{ret} dS \\
&+ \int_{f=0} \left[ \dfrac{L_r-L_M}{r^2(1-M_r)^2} \right]_{ret} dS
+ \dfrac{1}{c} \int_{f=0} \left[ \dfrac{L_r (r \dot{M_r} + c(M_r-M^2))}{r^2(1-M_r)^3} \right]_{ret} dS
\end{align*}
where $L_M = \vec{L} \cdot \vec{M}$.

\end{document}
