\documentclass[onecolumn,10pt]{jhwhw}

\usepackage{epsfig} %% for loading postscript figures
\usepackage{amsmath}
\usepackage{graphicx}
\usepackage{grffile}
\usepackage{pdfpages}
\usepackage{algpseudocode}
\usepackage{wrapfig}
\usepackage{booktabs}
\usepackage{multicol}
\usepackage{cancel}
% \usepackage{geometry}
% \geometry{margin=0.75in}

% Default fixed font does not support bold face
\DeclareFixedFont{\ttb}{T1}{txtt}{bx}{n}{12} % for bold
\DeclareFixedFont{\ttm}{T1}{txtt}{m}{n}{12} % for normal

% Custom colors
\usepackage{color}
\usepackage{listings}
\usepackage{framed}
\usepackage{caption}
\usepackage{bm}
\captionsetup[lstlisting]{font={small,tt}}

\author{John Karasinski}
\title{EAE 298 Aeroacoustics \\ Fall Quarter 2016 \\ Homework \#4}

\begin{document}
\maketitle

You are designing a new aircraft engine and analyzing acoustic propagation generated by a non-uniform flow with angle of attack interacting with rotating fans. You obtained flow fields from CFD for a one-sixth small scale of the engine. The radius for the small scale engine is 13 in and the hub radius is 3 in. CFD provides the velocity gust information as a function of its circumferential modes. The circumferential mode for acoustics can be expressed as $m= nB - kV$ where $B$ is the number of blades, $V$ is the number of vanes, $n$ stands for the harmonic of BPF and $k$ is the integer (1, 2, 3$\ldots$). You consider only positive k at this time (this is related the rotation direction of gust). The number of blades is 18. The number of vanes is considered to be 1 since there is no physical vanes but there is one revolution difference. The Mach number is 0.525 and the fan RPM is 8326.3042, the speed of sound is 13503.937009 in/s and the density is 1.4988E-5 slug/in$^3$. The dominant noise is generated at the 1st BPF or n=1 in which the angular frequency is given as $RPM \times \frac{2 \pi}{60} \times B$. We are interested in the propagation through the inlet of the engine so that sound propagates to -z direction assuming the +z direction is in the flow direction.

\problem{[20 points]}
Determine the first five eigenvalues of acoustics for m=18, 17, 16, 15 or (k=1, 2, 3, 4) or (m,n) = (18,0), (18,1), (18,2), (18,3), (18,4), (17,0), (17,1), (17,2), (17,3), (17,4), (16,0), (16,1), (16,2), (16,3), (16,4), (15,0), (15,1), (15,2), (15,3), (15,4)

\problem{[20 points]}
Plot the five eigenfunctions (radial modes, n=0, 1, 2, 3, 4) for m=18, 17, 16, 15 or (k=1, 2, 3, 4) and verify n describes the number of zero crossings in the radial direction

\problem{[20 points]}
Determine the wavenumbers in the z direction for (m,n)=(18,0), (18,1), (18,2), (17,0), (17,1), (17,2), (16,0), (16,1), (16,2), (15,0), (15,1), (15,2). Indicate whether the mode is cut-on (propagating) or cut-off (exponentially decaying). Consider only the propagation in the -z direction. Exclude the exponentially growing solution and include only the propagating solutions or exponentially decaying solutions.

\problem{[30 points]}
The pressure distribution file at z=0 plane for m=18 or 1 BPF is provided. The first column is the dimensional radius [in], the second column the real part of the pressure [psi], and the third column is the imaginary part of the pressure [psi]. Using this boundary condition, compute the sound power level for (m,n)=(18,0), (18,1), (18,2). This noise is considered for blade self noise that is not associated with the gust response since k=0. Note that the z=0 plane is not the same as the engine inlet. Use the conversion for the unit for the sound power as follows: PWL (dB)=10*log10((Wmn))-10*log10(7.3756E-13)

\problem{[10 points]}
The imaginary value of $k_z$ indicates the decay of sound pressure. Using the imaginary values of $k_z$, compute sound power level at the engine inlet. The distance between z=0 plane and the engine inlet is 4 in.

\end{document}
